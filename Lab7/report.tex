\documentclass[a4paper, 12pt]{article}

\usepackage{graphicx}
\usepackage{caption}
\usepackage[section]{placeins}
\usepackage{fixltx2e}
\usepackage[page]{appendix}
\usepackage[margin=0.9in]{geometry}
\usepackage{amsmath}
\usepackage{listings}
\usepackage{xcolor}
\usepackage{hyperref}
\usepackage{cleveref}
\hypersetup{
	colorlinks=true,
	linkcolor=blue,
	filecolor=blue,
	citecolor=blue,
	urlcolor=magenta,
}

\definecolor{mGreen}{rgb}{0,0.6,0}
\definecolor{mGray}{rgb}{0.5,0.5,0.5}
\definecolor{mPurple}{rgb}{0.58,0,0.82}
\definecolor{backgroundColour}{rgb}{0.95,0.95,0.92}
\lstset{
	language=VHDL,                % choose the language of the code
	% language=C
	numbers=left,                   % where to put the line-numbers
	stepnumber=1,                   % the step between two line-numbers.        
	numbersep=5pt,                  % how far the line-numbers are from the code
	backgroundcolor=\color{backgroundColour},
	commentstyle=\color{mGreen},
	keywordstyle=\color{magenta},
	numberstyle=\tiny\color{mGray},
	stringstyle=\color{mPurple},
	basicstyle=\footnotesize,
	showspaces=false,               % show spaces adding particular underscores
	showstringspaces=false,         % underline spaces within strings
	showtabs=false,                 % show tabs within strings adding particular underscores
	tabsize=4,                      % sets default tabsize to 2 spaces
	captionpos=b,                   % sets the caption-position to bottom
	breaklines=true,                % sets automatic line breaking
	breakatwhitespace=true,         % sets if automatic breaks should only happen at whitespace
	title=\lstname,                 % show the filename of files included with \lstinputlisting;
	inputpath=code,
}

\graphicspath{{./pictures/}}

\title{ECEN302 Lab 7 - TCL scripts and Packaging IP}
\author{Joshua Benfell - 300433229}

\begin{document}
	\maketitle
	
	\section{Introduction}
		The primary objective of this lab was to wrap up a module as an IP Block. This was done so that an understanding was gained of how to compartmentalise and modularise the program written for an FPGA was acquired for the purpose of speeding up development time in future endeavors. 
	\section{Methodology}
		The module being packed up in this lab is a PWM sine wave generator. This module takes the first nine switches as inputs to provide a binary weighting to a unity sine wave. Flicking the switches on would increase the time it takes for a point on the output sine wave to be generated; this in turn lowers the frequency of the output sine wave. The AUD\_SD port of the module was used to turn off the low pass filter which stops the audio signal from outputting. The AUD\_PWM is the output port for the generated tone. 
		\par
		To set up the project a tcl script was run which performed all the connections that needed to be made in vivado to create the PWM audio master module. tcl scripts are a good option to use for saving projects as it recreates them with the same commands that vivado runs automatically as you make the project from the ground up. This allows easy version control and modularity in created components.
		\par
		Once it was created it was opened in vivado and any dependencies were updated to the latest versions to ensure that everything can be found. This module makes use of a direct digital synthesizer to generate a sine wave of a certain frequency as detailed above and then uses a converter to output a PWM equivalent.
		\par
		This was then uploaded to the FPGA to verify it's functionality, and then it was abstracted so that it was no longer specific to the board being used, but rather the chip being used. This provides greater flexibility to where the IP core can be applied. It was then packaged with the IP packager tool and categorized as a waveform synthesis module to make it easier to find in the future. This was then validated by creating a managed IP project and doing a test import.
		\par
		Now that an IP block has been created, it was imported into a Vivado project through the block design editor. In addition to this, the external ports and clock interface were added to the design so that the IP block could function correctly. Finally, it was downloaded to the FPGA after creating a new constraints file and the functionality was verified to be the same as the pre-packaged version.
		
	% \Urlmuskip=0mu plus 1mu\relax
	% \bibliography{bibliography}
	% \bibliographystyle{IEEEtran}

	% \begin{appendices}           
	% \end{appendices}
\end{document}