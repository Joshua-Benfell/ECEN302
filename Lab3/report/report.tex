\documentclass[a4paper, 12pt]{article}

\usepackage{graphicx}
\usepackage{caption}
\usepackage[section]{placeins}
\usepackage{fixltx2e}
\usepackage[page]{appendix}
\usepackage[margin=0.9in]{geometry}
\usepackage{amsmath}
\usepackage{listings}
\usepackage{xcolor}
\usepackage{hyperref}
\usepackage{cleveref}
\hypersetup{
	colorlinks=true,
	linkcolor=blue,
	filecolor=blue,
	citecolor=blue,
	urlcolor=magenta,
}

\definecolor{mGreen}{rgb}{0,0.6,0}
\definecolor{mGray}{rgb}{0.5,0.5,0.5}
\definecolor{mPurple}{rgb}{0.58,0,0.82}
\definecolor{backgroundColour}{rgb}{0.95,0.95,0.92}
\lstset{
	language=VHDL,                % choose the language of the code
	% language=C
	numbers=left,                   % where to put the line-numbers
	stepnumber=1,                   % the step between two line-numbers.        
	numbersep=5pt,                  % how far the line-numbers are from the code
	backgroundcolor=\color{backgroundColour},
	commentstyle=\color{mGreen},
	keywordstyle=\color{magenta},
	numberstyle=\tiny\color{mGray},
	stringstyle=\color{mPurple},
	basicstyle=\footnotesize,
	showspaces=false,               % show spaces adding particular underscores
	showstringspaces=false,         % underline spaces within strings
	showtabs=false,                 % show tabs within strings adding particular underscores
	tabsize=4,                      % sets default tabsize to 2 spaces
	captionpos=b,                   % sets the caption-position to bottom
	breaklines=true,                % sets automatic line breaking
	breakatwhitespace=true,         % sets if automatic breaks should only happen at whitespace
	title=\lstname,                 % show the filename of files included with \lstinputlisting;
	inputpath=code,
}

\graphicspath{{./pictures/}}

\title{ECEN302 Lab 3 - Functions, Procedures, and Test Benches}
\author{Joshua Benfell - 300433229}

\begin{document}
	\maketitle
	
	\section{Objectives}
		The primary objectives of this lab was to learn to develop reusable VHDL code with the inbuilt procedures and functions. We are to learn the differences between both constructs in VHDL and this will help us develop our thinking and understanding about sequential and procedural programming. We will be able to verify this learning when we have developed models for combinatorial logic using both functions and procedures and when we develop test benches to verify their functionality.
	\section{Methodology}
		\subsection{Introduction}
			In this lab we will be implementing models for various combinatorial logic circuits in VHDL. This will be done using the inbuilt procedures and functions. The functionality of each of the implemented models will be verified through the use of test benches. Finally a test bench will be used to generate a waveform. This will provide a deeper understanding of how each of the constructs work and the features that are provided with them.
		\subsection{Procedures}
			The first procedural model that was created was one that would add two 4 bit values into a sum nybble and a carry bit. To do this a procedure was created that took the two numbers as input and outputted a 5 bit number. Inside this procedure is a single line that adds the numbers together.
		\subsection{Functions}
		
		\subsection{Testbench}

		\subsection{Code}
			\subsubsection{1-1 Procedures Add two Values}
				\lstinputlisting{lab311.vhd}
				\lstinputlisting{lab311_tb.vhd}
			\subsubsection{1-2 Calc Even Parity}
				\lstinputlisting{lab312.vhd}
				\lstinputlisting{lab312_tb.vhd}
			\subsubsection{2-1 Functions Add Two Values}
				\lstinputlisting{lab321.vhd}
				\lstinputlisting{lab321_tb.vhd}
			\subsubsection{2-2 Calc Ones}
				\lstinputlisting{lab322.vhd}
				\lstinputlisting{lab322_tb.vhd}
			\subsubsection{3-1 RCA Test Bench}
				\lstinputlisting{lab331_tb.vhd}
			\subsubsection{3-2 Waveform Generator Testbench}
				\lstinputlisting{lab332_tb.vhd}




	\section{Questions}


	% \Urlmuskip=0mu plus 1mu\relax
	% \bibliography{bibliography}
	% \bibliographystyle{IEEEtran}

	% \begin{appendices}           
	% \end{appendices}
\end{document}